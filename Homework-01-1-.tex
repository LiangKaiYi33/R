% Options for packages loaded elsewhere
\PassOptionsToPackage{unicode}{hyperref}
\PassOptionsToPackage{hyphens}{url}
%
\documentclass[
]{article}
\usepackage{lmodern}
\usepackage{amssymb,amsmath}
\usepackage{ifxetex,ifluatex}
\ifnum 0\ifxetex 1\fi\ifluatex 1\fi=0 % if pdftex
  \usepackage[T1]{fontenc}
  \usepackage[utf8]{inputenc}
  \usepackage{textcomp} % provide euro and other symbols
\else % if luatex or xetex
  \usepackage{unicode-math}
  \defaultfontfeatures{Scale=MatchLowercase}
  \defaultfontfeatures[\rmfamily]{Ligatures=TeX,Scale=1}
\fi
% Use upquote if available, for straight quotes in verbatim environments
\IfFileExists{upquote.sty}{\usepackage{upquote}}{}
\IfFileExists{microtype.sty}{% use microtype if available
  \usepackage[]{microtype}
  \UseMicrotypeSet[protrusion]{basicmath} % disable protrusion for tt fonts
}{}
\makeatletter
\@ifundefined{KOMAClassName}{% if non-KOMA class
  \IfFileExists{parskip.sty}{%
    \usepackage{parskip}
  }{% else
    \setlength{\parindent}{0pt}
    \setlength{\parskip}{6pt plus 2pt minus 1pt}}
}{% if KOMA class
  \KOMAoptions{parskip=half}}
\makeatother
\usepackage{xcolor}
\IfFileExists{xurl.sty}{\usepackage{xurl}}{} % add URL line breaks if available
\IfFileExists{bookmark.sty}{\usepackage{bookmark}}{\usepackage{hyperref}}
\hypersetup{
  pdftitle={Homework 1},
  hidelinks,
  pdfcreator={LaTeX via pandoc}}
\urlstyle{same} % disable monospaced font for URLs
\usepackage[margin=1in]{geometry}
\usepackage{color}
\usepackage{fancyvrb}
\newcommand{\VerbBar}{|}
\newcommand{\VERB}{\Verb[commandchars=\\\{\}]}
\DefineVerbatimEnvironment{Highlighting}{Verbatim}{commandchars=\\\{\}}
% Add ',fontsize=\small' for more characters per line
\usepackage{framed}
\definecolor{shadecolor}{RGB}{248,248,248}
\newenvironment{Shaded}{\begin{snugshade}}{\end{snugshade}}
\newcommand{\AlertTok}[1]{\textcolor[rgb]{0.94,0.16,0.16}{#1}}
\newcommand{\AnnotationTok}[1]{\textcolor[rgb]{0.56,0.35,0.01}{\textbf{\textit{#1}}}}
\newcommand{\AttributeTok}[1]{\textcolor[rgb]{0.77,0.63,0.00}{#1}}
\newcommand{\BaseNTok}[1]{\textcolor[rgb]{0.00,0.00,0.81}{#1}}
\newcommand{\BuiltInTok}[1]{#1}
\newcommand{\CharTok}[1]{\textcolor[rgb]{0.31,0.60,0.02}{#1}}
\newcommand{\CommentTok}[1]{\textcolor[rgb]{0.56,0.35,0.01}{\textit{#1}}}
\newcommand{\CommentVarTok}[1]{\textcolor[rgb]{0.56,0.35,0.01}{\textbf{\textit{#1}}}}
\newcommand{\ConstantTok}[1]{\textcolor[rgb]{0.00,0.00,0.00}{#1}}
\newcommand{\ControlFlowTok}[1]{\textcolor[rgb]{0.13,0.29,0.53}{\textbf{#1}}}
\newcommand{\DataTypeTok}[1]{\textcolor[rgb]{0.13,0.29,0.53}{#1}}
\newcommand{\DecValTok}[1]{\textcolor[rgb]{0.00,0.00,0.81}{#1}}
\newcommand{\DocumentationTok}[1]{\textcolor[rgb]{0.56,0.35,0.01}{\textbf{\textit{#1}}}}
\newcommand{\ErrorTok}[1]{\textcolor[rgb]{0.64,0.00,0.00}{\textbf{#1}}}
\newcommand{\ExtensionTok}[1]{#1}
\newcommand{\FloatTok}[1]{\textcolor[rgb]{0.00,0.00,0.81}{#1}}
\newcommand{\FunctionTok}[1]{\textcolor[rgb]{0.00,0.00,0.00}{#1}}
\newcommand{\ImportTok}[1]{#1}
\newcommand{\InformationTok}[1]{\textcolor[rgb]{0.56,0.35,0.01}{\textbf{\textit{#1}}}}
\newcommand{\KeywordTok}[1]{\textcolor[rgb]{0.13,0.29,0.53}{\textbf{#1}}}
\newcommand{\NormalTok}[1]{#1}
\newcommand{\OperatorTok}[1]{\textcolor[rgb]{0.81,0.36,0.00}{\textbf{#1}}}
\newcommand{\OtherTok}[1]{\textcolor[rgb]{0.56,0.35,0.01}{#1}}
\newcommand{\PreprocessorTok}[1]{\textcolor[rgb]{0.56,0.35,0.01}{\textit{#1}}}
\newcommand{\RegionMarkerTok}[1]{#1}
\newcommand{\SpecialCharTok}[1]{\textcolor[rgb]{0.00,0.00,0.00}{#1}}
\newcommand{\SpecialStringTok}[1]{\textcolor[rgb]{0.31,0.60,0.02}{#1}}
\newcommand{\StringTok}[1]{\textcolor[rgb]{0.31,0.60,0.02}{#1}}
\newcommand{\VariableTok}[1]{\textcolor[rgb]{0.00,0.00,0.00}{#1}}
\newcommand{\VerbatimStringTok}[1]{\textcolor[rgb]{0.31,0.60,0.02}{#1}}
\newcommand{\WarningTok}[1]{\textcolor[rgb]{0.56,0.35,0.01}{\textbf{\textit{#1}}}}
\usepackage{graphicx,grffile}
\makeatletter
\def\maxwidth{\ifdim\Gin@nat@width>\linewidth\linewidth\else\Gin@nat@width\fi}
\def\maxheight{\ifdim\Gin@nat@height>\textheight\textheight\else\Gin@nat@height\fi}
\makeatother
% Scale images if necessary, so that they will not overflow the page
% margins by default, and it is still possible to overwrite the defaults
% using explicit options in \includegraphics[width, height, ...]{}
\setkeys{Gin}{width=\maxwidth,height=\maxheight,keepaspectratio}
% Set default figure placement to htbp
\makeatletter
\def\fps@figure{htbp}
\makeatother
\setlength{\emergencystretch}{3em} % prevent overfull lines
\providecommand{\tightlist}{%
  \setlength{\itemsep}{0pt}\setlength{\parskip}{0pt}}
\setcounter{secnumdepth}{-\maxdimen} % remove section numbering

\title{Homework 1}
\author{}
\date{\vspace{-2.5em}}

\begin{document}
\maketitle

\begin{enumerate}
\def\labelenumi{\arabic{enumi}.}
\tightlist
\item
  The Iowa data set iowa.csv is a toy example that summarises the yield
  of wheat (bushels per acre) for the state of Iowa between 1930-1962.
  In addition to yield, year, rainfall and temperature were recorded as
  the main predictors of yield.
\end{enumerate}

\begin{enumerate}
\def\labelenumi{\alph{enumi}.}
\tightlist
\item
  First, we need to load the data set into R using the command
  \texttt{read.csv()}. Use the help function to learn what arguments
  this function takes. Once you have the necessary input, load the data
  set into R and make it a data frame called \texttt{iowa.df}.
\end{enumerate}

\textbf{解}:

\begin{Shaded}
\begin{Highlighting}[]
\NormalTok{iowa.df<-}\KeywordTok{read.csv}\NormalTok{(}\StringTok{"data/iowa.csv"}\NormalTok{,}\DataTypeTok{header=}\NormalTok{T,}\DataTypeTok{sep=}\StringTok{';'}\NormalTok{)}
\end{Highlighting}
\end{Shaded}

\begin{enumerate}
\def\labelenumi{\alph{enumi}.}
\setcounter{enumi}{1}
\tightlist
\item
  How many rows and columns does \texttt{iowa.df} have?
\end{enumerate}

\textbf{解}:

\begin{Shaded}
\begin{Highlighting}[]
\KeywordTok{dim}\NormalTok{(iowa.df)}
\end{Highlighting}
\end{Shaded}

\begin{verbatim}
## [1] 33 10
\end{verbatim}

\begin{enumerate}
\def\labelenumi{\alph{enumi}.}
\setcounter{enumi}{2}
\tightlist
\item
  What are the names of the columns of \texttt{iowa.df}?
\end{enumerate}

\textbf{解}:

\begin{Shaded}
\begin{Highlighting}[]
\KeywordTok{names}\NormalTok{(iowa.df)}
\end{Highlighting}
\end{Shaded}

\begin{verbatim}
##  [1] "Year"  "Rain0" "Temp1" "Rain1" "Temp2" "Rain2" "Temp3" "Rain3" "Temp4"
## [10] "Yield"
\end{verbatim}

\begin{enumerate}
\def\labelenumi{\alph{enumi}.}
\setcounter{enumi}{3}
\tightlist
\item
  What is the value of row 5, column 7 of \texttt{iowa.df}?
\end{enumerate}

\textbf{解}:

\begin{Shaded}
\begin{Highlighting}[]
\NormalTok{iowa.df[}\DecValTok{5}\NormalTok{,}\DecValTok{7}\NormalTok{]}
\end{Highlighting}
\end{Shaded}

\begin{verbatim}
## [1] 79.7
\end{verbatim}

\begin{enumerate}
\def\labelenumi{\alph{enumi}.}
\setcounter{enumi}{4}
\tightlist
\item
  Display the second row of \texttt{iowa.df} in its entirety.
\end{enumerate}

\textbf{解}:

\begin{Shaded}
\begin{Highlighting}[]
\NormalTok{iowa.df[}\DecValTok{2}\NormalTok{,]}
\end{Highlighting}
\end{Shaded}

\begin{verbatim}
##   Year Rain0 Temp1 Rain1 Temp2 Rain2 Temp3 Rain3 Temp4 Yield
## 2 1931 14.76  57.5  3.83    75  2.72  77.2   3.3  72.6  32.9
\end{verbatim}

\begin{enumerate}
\def\labelenumi{\arabic{enumi}.}
\setcounter{enumi}{1}
\tightlist
\item
  Syntax and class-typing.
\end{enumerate}

\begin{enumerate}
\def\labelenumi{\alph{enumi}.}
\tightlist
\item
  For each of the following commands, either explain why they should be
  errors, or explain the non-erroneous result.
\end{enumerate}

\begin{verbatim}
vector1 <- c("5", "12", "7", "32")
max(vector1)
sort(vector1)
sum(vector1)
\end{verbatim}

\textbf{解}:由于vector1中的数据格式为字符型,所以只能进行对字符类型的操作。max(vector1)的结果为``7''。sort(vector1)的结果为``12''
``32'' ``5'' ``7''。而sum()只能对数字型数据才有效,所以将会报错。

\begin{enumerate}
\def\labelenumi{\alph{enumi}.}
\setcounter{enumi}{1}
\tightlist
\item
  For the next series of commands, either explain their results, or why
  they should produce errors.
\end{enumerate}

\begin{verbatim}
vector2 <- c("5",7,12)
vector2[2] + vector2[3]

dataframe3 <- data.frame(z1="5",z2=7,z3=12)
dataframe3[1,2] + dataframe3[1,3]

list4 <- list(z1="6", z2=42, z3="49", z4=126)
list4[[2]]+list4[[4]]
list4[2]+list4[4]
\end{verbatim}

\textbf{解}:在用c()赋值时,如果元素中含有字符型数据,则将所有非字符数据都转化为字符型,所以vector2=``5''
``7'' ``12'',而对于字符串是没有加法操作的。

在用data.frame()进行数据框赋值时,数据类型不变,所以{[}1,2{]}和{[}1,3{]}的数据类型都是数字类型,可以进行dataframe3{[}1,2{]}+dataframe3{[}1,3{]},结果为19。

在list数据中,用list{[}i{]}调用的是其第i个组分,仍然是一个列表,所以无法做加法,而list{[}{[}i{]}{]}访问的是第i个组分的元素值,如这里list4{[}{[}2{]}{]}就是元素42,所以可以进行加法,list4{[}{[}2{]}{]}+list{[}{[}4{]}{]}=168.

\begin{enumerate}
\def\labelenumi{\arabic{enumi}.}
\setcounter{enumi}{2}
\tightlist
\item
  Working with functions and operators.
\end{enumerate}

\begin{enumerate}
\def\labelenumi{\alph{enumi}.}
\tightlist
\item
  The colon operator will create a sequence of integers in order. It is
  a special case of the function \texttt{seq()} which you saw earlier in
  this assignment. Using the help command \texttt{?seq} to learn about
  the function, design an expression that will give you the sequence of
  numbers from 1 to 10000 in increments of 372. Design another that will
  give you a sequence between 1 and 10000 that is exactly 50 numbers in
  length.
\end{enumerate}

\textbf{解}:

\begin{Shaded}
\begin{Highlighting}[]
\KeywordTok{seq}\NormalTok{(}\DataTypeTok{from=}\DecValTok{1}\NormalTok{,}\DataTypeTok{to=}\DecValTok{10000}\NormalTok{,}\DataTypeTok{by=}\DecValTok{372}\NormalTok{)}
\end{Highlighting}
\end{Shaded}

\begin{verbatim}
##  [1]    1  373  745 1117 1489 1861 2233 2605 2977 3349 3721 4093 4465 4837 5209
## [16] 5581 5953 6325 6697 7069 7441 7813 8185 8557 8929 9301 9673
\end{verbatim}

\begin{Shaded}
\begin{Highlighting}[]
\KeywordTok{seq}\NormalTok{(}\DataTypeTok{from=}\DecValTok{1}\NormalTok{,}\DataTypeTok{to=}\DecValTok{10000}\NormalTok{,}\DataTypeTok{length.out=}\DecValTok{50}\NormalTok{)}
\end{Highlighting}
\end{Shaded}

\begin{verbatim}
##  [1]     1.0000   205.0612   409.1224   613.1837   817.2449  1021.3061
##  [7]  1225.3673  1429.4286  1633.4898  1837.5510  2041.6122  2245.6735
## [13]  2449.7347  2653.7959  2857.8571  3061.9184  3265.9796  3470.0408
## [19]  3674.1020  3878.1633  4082.2245  4286.2857  4490.3469  4694.4082
## [25]  4898.4694  5102.5306  5306.5918  5510.6531  5714.7143  5918.7755
## [31]  6122.8367  6326.8980  6530.9592  6735.0204  6939.0816  7143.1429
## [37]  7347.2041  7551.2653  7755.3265  7959.3878  8163.4490  8367.5102
## [43]  8571.5714  8775.6327  8979.6939  9183.7551  9387.8163  9591.8776
## [49]  9795.9388 10000.0000
\end{verbatim}

\begin{enumerate}
\def\labelenumi{\alph{enumi}.}
\setcounter{enumi}{1}
\tightlist
\item
  The function \texttt{rep()} repeats a vector some number of times.
  Explain the difference between `rep(1:3, times=3) and rep(1:3,
  each=3).
\end{enumerate}

\textbf{解}:前者是将序列整体重复3遍,结果是123123123;后者是将每个元素分别重复三遍,结果为111222333。

MB.Ch1.2. The orings data frame gives data on the damage that had
occurred in US space shuttle launches prior to the disastrous Challenger
launch of 28 January 1986. The observations in rows 1, 2, 4, 11, 13, and
18 were included in the pre-launch charts used in deciding whether to
proceed with the launch, while remaining rows were omitted.

Create a new data frame by extracting these rows from orings, and plot
total incidents against temperature for this new data frame. Obtain a
similar plot for the full data set.

\textbf{解}:

\begin{Shaded}
\begin{Highlighting}[]
\NormalTok{newdata1=orings[}\KeywordTok{c}\NormalTok{(}\DecValTok{1}\NormalTok{,}\DecValTok{2}\NormalTok{,}\DecValTok{4}\NormalTok{,}\DecValTok{11}\NormalTok{,}\DecValTok{13}\NormalTok{),]}
\KeywordTok{plot}\NormalTok{(Total}\OperatorTok{~}\NormalTok{Temperature,}\DataTypeTok{data=}\NormalTok{newdata1,}\DataTypeTok{col=}\DecValTok{2}\NormalTok{,}\DataTypeTok{pch=}\DecValTok{16}\NormalTok{,}\DataTypeTok{cex=}\DecValTok{1}\NormalTok{)}
\end{Highlighting}
\end{Shaded}

\includegraphics{Homework-01-1-_files/figure-latex/unnamed-chunk-7-1.pdf}

MB.Ch1.4. For the data frame ais (DAAG package)

\begin{enumerate}
\def\labelenumi{(\alph{enumi})}
\tightlist
\item
  Use the function str() to get information on each of the columns.
  Determine whether any of the columns hold missing values.
\end{enumerate}

\textbf{解}:

\begin{Shaded}
\begin{Highlighting}[]
\KeywordTok{str}\NormalTok{(ais)}
\end{Highlighting}
\end{Shaded}

\begin{verbatim}
## 'data.frame':    202 obs. of  13 variables:
##  $ rcc   : num  3.96 4.41 4.14 4.11 4.45 4.1 4.31 4.42 4.3 4.51 ...
##  $ wcc   : num  7.5 8.3 5 5.3 6.8 4.4 5.3 5.7 8.9 4.4 ...
##  $ hc    : num  37.5 38.2 36.4 37.3 41.5 37.4 39.6 39.9 41.1 41.6 ...
##  $ hg    : num  12.3 12.7 11.6 12.6 14 12.5 12.8 13.2 13.5 12.7 ...
##  $ ferr  : num  60 68 21 69 29 42 73 44 41 44 ...
##  $ bmi   : num  20.6 20.7 21.9 21.9 19 ...
##  $ ssf   : num  109.1 102.8 104.6 126.4 80.3 ...
##  $ pcBfat: num  19.8 21.3 19.9 23.7 17.6 ...
##  $ lbm   : num  63.3 58.5 55.4 57.2 53.2 ...
##  $ ht    : num  196 190 178 185 185 ...
##  $ wt    : num  78.9 74.4 69.1 74.9 64.6 63.7 75.2 62.3 66.5 62.9 ...
##  $ sex   : Factor w/ 2 levels "f","m": 1 1 1 1 1 1 1 1 1 1 ...
##  $ sport : Factor w/ 10 levels "B_Ball","Field",..: 1 1 1 1 1 1 1 1 1 1 ...
\end{verbatim}

\begin{Shaded}
\begin{Highlighting}[]
\ControlFlowTok{for}\NormalTok{(i }\ControlFlowTok{in}\NormalTok{ (}\DecValTok{1}\OperatorTok{:}\KeywordTok{dim}\NormalTok{(ais)[}\DecValTok{2}\NormalTok{]))\{}
\NormalTok{    vec<-}\KeywordTok{is.na}\NormalTok{(ais[,i])}
    \ControlFlowTok{if}\NormalTok{(}\KeywordTok{any}\NormalTok{(vec))\{}
        \KeywordTok{cat}\NormalTok{(}\StringTok{"the column"}\NormalTok{,i,}\StringTok{"hold missing values"}\NormalTok{)}
\NormalTok{    \}}
\NormalTok{\}}
\end{Highlighting}
\end{Shaded}

结果没有输出显示,所以没有列含有NA缺省值。

\begin{enumerate}
\def\labelenumi{(\alph{enumi})}
\setcounter{enumi}{1}
\tightlist
\item
  Make a table that shows the numbers of males and females for each
  different sport. In which sports is there a large imbalance (e.g., by
  a factor of more than 2:1) in the numbers of the two sexes?
\end{enumerate}

\textbf{解}:

\begin{Shaded}
\begin{Highlighting}[]
\KeywordTok{table}\NormalTok{(}\DataTypeTok{Sex=}\NormalTok{ais}\OperatorTok{$}\NormalTok{sex,}\DataTypeTok{Sports=}\NormalTok{ais}\OperatorTok{$}\NormalTok{sport)}
\end{Highlighting}
\end{Shaded}

\begin{verbatim}
##    Sports
## Sex B_Ball Field Gym Netball Row Swim T_400m T_Sprnt Tennis W_Polo
##   f     13     7   4      23  22    9     11       4      7      0
##   m     12    12   0       0  15   13     18      11      4     17
\end{verbatim}

其中Gym,Netball,T\_Sprnt,W\_polo男女相差比较大。

MB.Ch1.6.Create a data frame called Manitoba.lakes that contains the
lake's elevation (in meters above sea level) and area (in square
kilometers) as listed below. Assign the names of the lakes using the
row.names() function.

\textbf{解}:

\begin{Shaded}
\begin{Highlighting}[]
\NormalTok{ele<-}\KeywordTok{c}\NormalTok{(}\DecValTok{217}\NormalTok{,}\DecValTok{254}\NormalTok{,}\DecValTok{248}\NormalTok{,}\DecValTok{254}\NormalTok{,}\DecValTok{253}\NormalTok{,}\DecValTok{227}\NormalTok{,}\DecValTok{178}\NormalTok{,}\DecValTok{207}\NormalTok{,}\DecValTok{217}\NormalTok{)}
\NormalTok{are<-}\KeywordTok{c}\NormalTok{(}\DecValTok{24387}\NormalTok{,}\DecValTok{5374}\NormalTok{,}\DecValTok{4624}\NormalTok{,}\DecValTok{2247}\NormalTok{,}\DecValTok{1353}\NormalTok{,}\DecValTok{1223}\NormalTok{,}\DecValTok{1151}\NormalTok{,}\DecValTok{755}\NormalTok{,}\DecValTok{657}\NormalTok{)}
\NormalTok{Manitoba.lakes<-}\KeywordTok{data.frame}\NormalTok{(}\DataTypeTok{elevation=}\NormalTok{ele,}\DataTypeTok{area=}\NormalTok{are)}
\KeywordTok{row.names}\NormalTok{(Manitoba.lakes)<-}\KeywordTok{c}\NormalTok{(}\StringTok{"Winnipeg"}\NormalTok{,}\StringTok{"Winnipegosis"}\NormalTok{,}\StringTok{"Manitoba"}\NormalTok{,}\StringTok{"SouthernIndian"}\NormalTok{,}\StringTok{"Cedar"}\NormalTok{,}\StringTok{"Island"}\NormalTok{,}\StringTok{"Gods"}\NormalTok{,}\StringTok{"Cross"}\NormalTok{,}\StringTok{"Playgreen"}\NormalTok{)}
\NormalTok{Manitoba.lakes}
\end{Highlighting}
\end{Shaded}

\begin{verbatim}
##                elevation  area
## Winnipeg             217 24387
## Winnipegosis         254  5374
## Manitoba             248  4624
## SouthernIndian       254  2247
## Cedar                253  1353
## Island               227  1223
## Gods                 178  1151
## Cross                207   755
## Playgreen            217   657
\end{verbatim}

\begin{enumerate}
\def\labelenumi{(\alph{enumi})}
\tightlist
\item
  Use the following code to plot log2(area) versus elevation, adding
  labeling infor- mation (there is an extreme value of area that makes a
  logarithmic scale pretty much essential):
\end{enumerate}

\textbf{解}:

\begin{Shaded}
\begin{Highlighting}[]
\KeywordTok{attach}\NormalTok{(Manitoba.lakes)}
\KeywordTok{plot}\NormalTok{(}\KeywordTok{log2}\NormalTok{(area) }\OperatorTok{~}\StringTok{ }\NormalTok{elevation, }\DataTypeTok{pch=}\DecValTok{16}\NormalTok{, }\DataTypeTok{xlim=}\KeywordTok{c}\NormalTok{(}\DecValTok{170}\NormalTok{,}\DecValTok{280}\NormalTok{))}
\CommentTok{# NB: Doubling the area increases log2(area) by 1.0}
\KeywordTok{text}\NormalTok{(}\KeywordTok{log2}\NormalTok{(area) }\OperatorTok{~}\StringTok{ }\NormalTok{elevation, }\DataTypeTok{labels=}\KeywordTok{row.names}\NormalTok{(Manitoba.lakes), }\DataTypeTok{pos=}\DecValTok{4}\NormalTok{)}
\KeywordTok{text}\NormalTok{(}\KeywordTok{log2}\NormalTok{(area) }\OperatorTok{~}\StringTok{ }\NormalTok{elevation, }\DataTypeTok{labels=}\NormalTok{area, }\DataTypeTok{pos=}\DecValTok{2}\NormalTok{) }
\KeywordTok{title}\NormalTok{(}\DataTypeTok{main=}\StringTok{"Manitoba’s Largest Lakes"}\NormalTok{,}\DataTypeTok{sub=}\StringTok{"use the logarithmic to smooth the extreme value"}\NormalTok{)}
\end{Highlighting}
\end{Shaded}

\includegraphics{Homework-01-1-_files/figure-latex/unnamed-chunk-11-1.pdf}

Devise captions that explain the labeling on the points and on the
y-axis. It will be necessary to explain how distances on the scale
relate to changes in area.

\begin{enumerate}
\def\labelenumi{(\alph{enumi})}
\setcounter{enumi}{1}
\tightlist
\item
  Repeat the plot and associated labeling, now plotting area versus
  elevation, but specifying log=``y'' in order to obtain a logarithmic
  y-scale.
\end{enumerate}

\textbf{解}:

\begin{Shaded}
\begin{Highlighting}[]
\KeywordTok{plot.default}\NormalTok{(area }\OperatorTok{~}\StringTok{ }\NormalTok{elevation, }\DataTypeTok{pch=}\DecValTok{16}\NormalTok{, }\DataTypeTok{xlim=}\KeywordTok{c}\NormalTok{(}\DecValTok{170}\NormalTok{,}\DecValTok{280}\NormalTok{),}\DataTypeTok{log=}\StringTok{"y"}\NormalTok{)}
\KeywordTok{text.default}\NormalTok{(area }\OperatorTok{~}\StringTok{ }\NormalTok{elevation, }\DataTypeTok{labels=}\KeywordTok{row.names}\NormalTok{(Manitoba.lakes), }\DataTypeTok{pos=}\DecValTok{4}\NormalTok{)}
\KeywordTok{text.default}\NormalTok{(area }\OperatorTok{~}\StringTok{ }\NormalTok{elevation, }\DataTypeTok{labels=}\NormalTok{area, }\DataTypeTok{pos=}\DecValTok{2}\NormalTok{) }
\KeywordTok{title}\NormalTok{(}\StringTok{"Manitoba’s Largest Lakes"}\NormalTok{)}
\end{Highlighting}
\end{Shaded}

\includegraphics{Homework-01-1-_files/figure-latex/unnamed-chunk-12-1.pdf}

MB.Ch1.7. Look up the help page for the R function dotchart(). Use this
function to display the areas of the Manitoba lakes (a) on a linear
scale, and (b) on a logarithmic scale. Add, in each case, suitable
labeling information.

\textbf{解}: (a)

\begin{Shaded}
\begin{Highlighting}[]
\KeywordTok{dotchart}\NormalTok{(area,}\KeywordTok{row.names}\NormalTok{(Manitoba.lakes))}
\end{Highlighting}
\end{Shaded}

\includegraphics{Homework-01-1-_files/figure-latex/unnamed-chunk-13-1.pdf}

(b)

\begin{Shaded}
\begin{Highlighting}[]
\KeywordTok{dotchart}\NormalTok{(area,}\KeywordTok{row.names}\NormalTok{(Manitoba.lakes),}\DataTypeTok{log=}\StringTok{"x"}\NormalTok{)}
\end{Highlighting}
\end{Shaded}

\includegraphics{Homework-01-1-_files/figure-latex/unnamed-chunk-14-1.pdf}

MB.Ch1.8. Using the sum() function, obtain a lower bound for the area of
Manitoba covered by water.

\textbf{解}:

\begin{Shaded}
\begin{Highlighting}[]
\KeywordTok{sum}\NormalTok{(area)}
\end{Highlighting}
\end{Shaded}

\begin{verbatim}
## [1] 41771
\end{verbatim}

\end{document}
